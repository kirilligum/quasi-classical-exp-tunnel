%%fakesection
\documentclass{article}
\usepackage[usenames,dvipsnames]{color}
\usepackage{soul}
\usepackage{fullpage}
\usepackage{xcolor}
\usepackage{hyperref}
\usepackage{pifont}
\usepackage{gnuplottex}
\usepackage{float}


\let\Item\item
\renewcommand\item{\normalcolor\Item}

\newcommand\ml{\color[RGB]{153, 150, 204}} %may be later
\newcommand\later{\color[RGB]{153, 204, 150}} %later
\newcommand\nn{\color[RGB]{124, 124, 255}} %no need to do
\newcommand\done{\color[RGB]{129, 180, 185} \ding{52} }
\newcommand\now{\color[RGB]{255, 0, 0}} %current

\begin{document}

\section{exp-to-exp tunneling}

\subsection{Idea}
The idea is to model a tunneling from increasing exponential to the decreasing one with a gaussian coupling. 

\begin{equation}
  H = \frac{p^2}{2m} + e^r n1 + e^{-r} n2  + e^{-r^2} n1 n2
\end{equation}

plot here
\begin{figure}[h]
  \centering
  \begin{gnuplot}
    plot [-2:2] exp(x) lw 2 , exp(-x) lw 2, 2*exp(-x*x) lw 2 
  \end{gnuplot}
\end{figure}

\subsection{Problem}

The standard binning procedure that we used before doesn't work.

Suppose the initial condition is a gaussian distribution around $p=10$, $q=-10$.
If $n1=1$ and $n2=0$, everything works. However, when we start using binning, $n2\neq 0$.
This means that the term $e^{-r}n2$ can be very large and dominate the hamiltonian. 
The same problem appears after the transition through the barrier since $n1\neq0$.

\subsection{Proposed solutions}

There are 3 ways to solve the problem by changing:
\begin{enumerate}
  \item the potential -- similar to Tully
  \item the sampling
  \item varience in $n$ 
\end{enumerate}

\subsection{Changing the potential}
\subsection{Proportional sampling}
we need to change the sampling so that in the begining and in the end the energy doesn't go infinite.
to do that, we change the sizes of the sampling bins relative to the size of the potential. we also throw away trajectories after the tunneling
when the traj goes outside of the bin+bin\_varience, it gets thrown out
\subsection{Dynamic variance}
v1*n1 becomes something like v1*(n1i + (n1-n1i)/v1) or v1*(n1i + (n1-n1i)*exp(v1-v2)) or v1/v2.
the idea is to make the bin for n1 smaller when one of the potentials is much larger then the other. 



\section{keys}
\begin{itemize}
  \item \ml may be later
  \item \done done
  \item \nn no need to do
  \item \now current 
\end{itemize}

t[...] for tags

t[priorety unsorted] or t[pu]-- not sorted in priorety yet

\section{doing}
\subsection{potential-based binning for many traj}
\section{to do}
\subsection{change the hamiltonian to have dynamic variation of ns}
\subsection{change the potential to tully's}
\subsection{think of another way to change the potential}
\section{to do priority unsorted}
\section{plan}
%%%%%%%%%%%%%%%%%%
\section{done}
\subsection{when the increasing potential comes too high, kill the bins}
Sun Sep  1 01:15:35 PDT 2013
\begin{figure}[h]
  \centering
  % GNUPLOT: LaTeX picture with Postscript
\begingroup
  \makeatletter
  \providecommand\color[2][]{%
    \GenericError{(gnuplot) \space\space\space\@spaces}{%
      Package color not loaded in conjunction with
      terminal option `colourtext'%
    }{See the gnuplot documentation for explanation.%
    }{Either use 'blacktext' in gnuplot or load the package
      color.sty in LaTeX.}%
    \renewcommand\color[2][]{}%
  }%
  \providecommand\includegraphics[2][]{%
    \GenericError{(gnuplot) \space\space\space\@spaces}{%
      Package graphicx or graphics not loaded%
    }{See the gnuplot documentation for explanation.%
    }{The gnuplot epslatex terminal needs graphicx.sty or graphics.sty.}%
    \renewcommand\includegraphics[2][]{}%
  }%
  \providecommand\rotatebox[2]{#2}%
  \@ifundefined{ifGPcolor}{%
    \newif\ifGPcolor
    \GPcolorfalse
  }{}%
  \@ifundefined{ifGPblacktext}{%
    \newif\ifGPblacktext
    \GPblacktexttrue
  }{}%
  % define a \g@addto@macro without @ in the name:
  \let\gplgaddtomacro\g@addto@macro
  % define empty templates for all commands taking text:
  \gdef\gplbacktext{}%
  \gdef\gplfronttext{}%
  \makeatother
  \ifGPblacktext
    % no textcolor at all
    \def\colorrgb#1{}%
    \def\colorgray#1{}%
  \else
    % gray or color?
    \ifGPcolor
      \def\colorrgb#1{\color[rgb]{#1}}%
      \def\colorgray#1{\color[gray]{#1}}%
      \expandafter\def\csname LTw\endcsname{\color{white}}%
      \expandafter\def\csname LTb\endcsname{\color{black}}%
      \expandafter\def\csname LTa\endcsname{\color{black}}%
      \expandafter\def\csname LT0\endcsname{\color[rgb]{1,0,0}}%
      \expandafter\def\csname LT1\endcsname{\color[rgb]{0,1,0}}%
      \expandafter\def\csname LT2\endcsname{\color[rgb]{0,0,1}}%
      \expandafter\def\csname LT3\endcsname{\color[rgb]{1,0,1}}%
      \expandafter\def\csname LT4\endcsname{\color[rgb]{0,1,1}}%
      \expandafter\def\csname LT5\endcsname{\color[rgb]{1,1,0}}%
      \expandafter\def\csname LT6\endcsname{\color[rgb]{0,0,0}}%
      \expandafter\def\csname LT7\endcsname{\color[rgb]{1,0.3,0}}%
      \expandafter\def\csname LT8\endcsname{\color[rgb]{0.5,0.5,0.5}}%
    \else
      % gray
      \def\colorrgb#1{\color{black}}%
      \def\colorgray#1{\color[gray]{#1}}%
      \expandafter\def\csname LTw\endcsname{\color{white}}%
      \expandafter\def\csname LTb\endcsname{\color{black}}%
      \expandafter\def\csname LTa\endcsname{\color{black}}%
      \expandafter\def\csname LT0\endcsname{\color{black}}%
      \expandafter\def\csname LT1\endcsname{\color{black}}%
      \expandafter\def\csname LT2\endcsname{\color{black}}%
      \expandafter\def\csname LT3\endcsname{\color{black}}%
      \expandafter\def\csname LT4\endcsname{\color{black}}%
      \expandafter\def\csname LT5\endcsname{\color{black}}%
      \expandafter\def\csname LT6\endcsname{\color{black}}%
      \expandafter\def\csname LT7\endcsname{\color{black}}%
      \expandafter\def\csname LT8\endcsname{\color{black}}%
    \fi
  \fi
  \setlength{\unitlength}{0.0500bp}%
  \begin{picture}(7200.00,5040.00)%
    \gplgaddtomacro\gplbacktext{%
      \csname LTb\endcsname%
      \put(594,440){\makebox(0,0)[r]{\strut{}-6}}%
      \put(594,874){\makebox(0,0)[r]{\strut{}-4}}%
      \put(594,1307){\makebox(0,0)[r]{\strut{}-2}}%
      \put(594,1741){\makebox(0,0)[r]{\strut{} 0}}%
      \put(594,2174){\makebox(0,0)[r]{\strut{} 2}}%
      \put(594,2608){\makebox(0,0)[r]{\strut{} 4}}%
      \put(594,3041){\makebox(0,0)[r]{\strut{} 6}}%
      \put(594,3475){\makebox(0,0)[r]{\strut{} 8}}%
      \put(594,3908){\makebox(0,0)[r]{\strut{} 10}}%
      \put(594,4342){\makebox(0,0)[r]{\strut{} 12}}%
      \put(594,4775){\makebox(0,0)[r]{\strut{} 14}}%
      \put(726,220){\makebox(0,0){\strut{} 0}}%
      \put(1486,220){\makebox(0,0){\strut{} 2}}%
      \put(2245,220){\makebox(0,0){\strut{} 4}}%
      \put(3005,220){\makebox(0,0){\strut{} 6}}%
      \put(3765,220){\makebox(0,0){\strut{} 8}}%
      \put(4524,220){\makebox(0,0){\strut{} 10}}%
      \put(5284,220){\makebox(0,0){\strut{} 12}}%
      \put(6043,220){\makebox(0,0){\strut{} 14}}%
      \put(6803,220){\makebox(0,0){\strut{} 16}}%
    }%
    \gplgaddtomacro\gplfronttext{%
      \csname LTb\endcsname%
      \put(5816,4602){\makebox(0,0)[r]{\strut{}"traj.dat" u 1:2}}%
      \csname LTb\endcsname%
      \put(5816,4382){\makebox(0,0)[r]{\strut{}"" u 1:3}}%
      \csname LTb\endcsname%
      \put(5816,4162){\makebox(0,0)[r]{\strut{}"" u 1:8}}%
      \csname LTb\endcsname%
      \put(5816,3942){\makebox(0,0)[r]{\strut{}"" u 1:9}}%
    }%
    \gplbacktext
    \put(0,0){\includegraphics{endpotbin}}%
    \gplfronttext
  \end{picture}%
\endgroup

  \caption{when v1 becomes too large, we cut n1 to 0 and n2 to 1}
  \label{fig:endpotbin}
\end{figure}
\subsection{switch off the dynamics when potential comes to the same level as before}
Sun Sep  1 01:14:06 PDT 2013

didn't work that well. it switches too late
\subsection{proportional binning in the begining}
Sat Aug 31 14:57:57 PDT 2013
\begin{itemize}
  \item \done put v1 and v2 potentials into methods
  \item \done calculate the energy of the surfaces from the initial state before binning.
  \item \done change the bins proportional to the energies
\end{itemize}
Sat Aug 31 14:51:59 PDT 2013
\begin{figure}[H]
  \centering
  % GNUPLOT: LaTeX picture with Postscript
\begingroup
  \makeatletter
  \providecommand\color[2][]{%
    \GenericError{(gnuplot) \space\space\space\@spaces}{%
      Package color not loaded in conjunction with
      terminal option `colourtext'%
    }{See the gnuplot documentation for explanation.%
    }{Either use 'blacktext' in gnuplot or load the package
      color.sty in LaTeX.}%
    \renewcommand\color[2][]{}%
  }%
  \providecommand\includegraphics[2][]{%
    \GenericError{(gnuplot) \space\space\space\@spaces}{%
      Package graphicx or graphics not loaded%
    }{See the gnuplot documentation for explanation.%
    }{The gnuplot epslatex terminal needs graphicx.sty or graphics.sty.}%
    \renewcommand\includegraphics[2][]{}%
  }%
  \providecommand\rotatebox[2]{#2}%
  \@ifundefined{ifGPcolor}{%
    \newif\ifGPcolor
    \GPcolorfalse
  }{}%
  \@ifundefined{ifGPblacktext}{%
    \newif\ifGPblacktext
    \GPblacktexttrue
  }{}%
  % define a \g@addto@macro without @ in the name:
  \let\gplgaddtomacro\g@addto@macro
  % define empty templates for all commands taking text:
  \gdef\gplbacktext{}%
  \gdef\gplfronttext{}%
  \makeatother
  \ifGPblacktext
    % no textcolor at all
    \def\colorrgb#1{}%
    \def\colorgray#1{}%
  \else
    % gray or color?
    \ifGPcolor
      \def\colorrgb#1{\color[rgb]{#1}}%
      \def\colorgray#1{\color[gray]{#1}}%
      \expandafter\def\csname LTw\endcsname{\color{white}}%
      \expandafter\def\csname LTb\endcsname{\color{black}}%
      \expandafter\def\csname LTa\endcsname{\color{black}}%
      \expandafter\def\csname LT0\endcsname{\color[rgb]{1,0,0}}%
      \expandafter\def\csname LT1\endcsname{\color[rgb]{0,1,0}}%
      \expandafter\def\csname LT2\endcsname{\color[rgb]{0,0,1}}%
      \expandafter\def\csname LT3\endcsname{\color[rgb]{1,0,1}}%
      \expandafter\def\csname LT4\endcsname{\color[rgb]{0,1,1}}%
      \expandafter\def\csname LT5\endcsname{\color[rgb]{1,1,0}}%
      \expandafter\def\csname LT6\endcsname{\color[rgb]{0,0,0}}%
      \expandafter\def\csname LT7\endcsname{\color[rgb]{1,0.3,0}}%
      \expandafter\def\csname LT8\endcsname{\color[rgb]{0.5,0.5,0.5}}%
    \else
      % gray
      \def\colorrgb#1{\color{black}}%
      \def\colorgray#1{\color[gray]{#1}}%
      \expandafter\def\csname LTw\endcsname{\color{white}}%
      \expandafter\def\csname LTb\endcsname{\color{black}}%
      \expandafter\def\csname LTa\endcsname{\color{black}}%
      \expandafter\def\csname LT0\endcsname{\color{black}}%
      \expandafter\def\csname LT1\endcsname{\color{black}}%
      \expandafter\def\csname LT2\endcsname{\color{black}}%
      \expandafter\def\csname LT3\endcsname{\color{black}}%
      \expandafter\def\csname LT4\endcsname{\color{black}}%
      \expandafter\def\csname LT5\endcsname{\color{black}}%
      \expandafter\def\csname LT6\endcsname{\color{black}}%
      \expandafter\def\csname LT7\endcsname{\color{black}}%
      \expandafter\def\csname LT8\endcsname{\color{black}}%
    \fi
  \fi
  \setlength{\unitlength}{0.0500bp}%
  \begin{picture}(7200.00,5040.00)%
    \gplgaddtomacro\gplbacktext{%
      \csname LTb\endcsname%
      \put(594,440){\makebox(0,0)[r]{\strut{}-10}}%
      \put(594,1163){\makebox(0,0)[r]{\strut{}-5}}%
      \put(594,1885){\makebox(0,0)[r]{\strut{} 0}}%
      \put(594,2608){\makebox(0,0)[r]{\strut{} 5}}%
      \put(594,3330){\makebox(0,0)[r]{\strut{} 10}}%
      \put(594,4053){\makebox(0,0)[r]{\strut{} 15}}%
      \put(594,4775){\makebox(0,0)[r]{\strut{} 20}}%
      \put(726,220){\makebox(0,0){\strut{} 0}}%
      \put(2245,220){\makebox(0,0){\strut{} 5}}%
      \put(3765,220){\makebox(0,0){\strut{} 10}}%
      \put(5284,220){\makebox(0,0){\strut{} 15}}%
      \put(6803,220){\makebox(0,0){\strut{} 20}}%
    }%
    \gplgaddtomacro\gplfronttext{%
      \csname LTb\endcsname%
      \put(5816,4602){\makebox(0,0)[r]{\strut{}"traj.dat" u 1:2}}%
      \csname LTb\endcsname%
      \put(5816,4382){\makebox(0,0)[r]{\strut{}"" u 1:3}}%
      \csname LTb\endcsname%
      \put(5816,4162){\makebox(0,0)[r]{\strut{}"" u 1:8}}%
      \csname LTb\endcsname%
      \put(5816,3942){\makebox(0,0)[r]{\strut{}"" u 1:9}}%
    }%
    \gplbacktext
    \put(0,0){\includegraphics{proportional_binning}}%
    \gplfronttext
  \end{picture}%
\endgroup

  \caption{in the begining bin size is porprtional to the surface's energy and sum to 1}
  \label{fig:proportional_binning}
\end{figure}
when both momentum and poition is positive, and n1 fully transfered to n2, the particle should keep moving. 
that is when i should make n1 = 0.
\subsection{mark thrown out trajectory as bad}
Fri Aug 30 19:29:01 PDT 2013
in the state vector, after energy 0 is for a good traj, 1 is for bad. it's a bit of a waste to have it as a double but it's ok.
Fri Aug 30 18:41:58 PDT 2013
\subsection{make a function to throw out trajs that get way out of range}
\begin{itemize}
  \item give initial energy as a parameter.
  \item when the deviation of the energy becomes 2x large, throw the trajectory out
\end{itemize}
\subsection{try on short scale with low energies}
Fri Aug 30 09:09:53 PDT 2013
\subsection{make different bins for n1 and n2}
Thu Aug 29 17:06:06 PDT 2013
\subsection{pass initial n1 and n2 to the hamiltonian}
Thu Aug 29 14:20:20 PDT 2013
\subsection{clean parameters a bit}
Thu Aug 29 13:41:59 PDT 2013
\subsection{simplify parameters so the numbers are around 1}
Thu Aug 29 14:17:04 PDT 2013
\section{notes}
\subsection{short scale first tunneling}
Fri Aug 30 08:53:57 PDT 2013
\begin{figure}[H]
  \centering
  % GNUPLOT: LaTeX picture with Postscript
\begingroup
  \makeatletter
  \providecommand\color[2][]{%
    \GenericError{(gnuplot) \space\space\space\@spaces}{%
      Package color not loaded in conjunction with
      terminal option `colourtext'%
    }{See the gnuplot documentation for explanation.%
    }{Either use 'blacktext' in gnuplot or load the package
      color.sty in LaTeX.}%
    \renewcommand\color[2][]{}%
  }%
  \providecommand\includegraphics[2][]{%
    \GenericError{(gnuplot) \space\space\space\@spaces}{%
      Package graphicx or graphics not loaded%
    }{See the gnuplot documentation for explanation.%
    }{The gnuplot epslatex terminal needs graphicx.sty or graphics.sty.}%
    \renewcommand\includegraphics[2][]{}%
  }%
  \providecommand\rotatebox[2]{#2}%
  \@ifundefined{ifGPcolor}{%
    \newif\ifGPcolor
    \GPcolorfalse
  }{}%
  \@ifundefined{ifGPblacktext}{%
    \newif\ifGPblacktext
    \GPblacktexttrue
  }{}%
  % define a \g@addto@macro without @ in the name:
  \let\gplgaddtomacro\g@addto@macro
  % define empty templates for all commands taking text:
  \gdef\gplbacktext{}%
  \gdef\gplfronttext{}%
  \makeatother
  \ifGPblacktext
    % no textcolor at all
    \def\colorrgb#1{}%
    \def\colorgray#1{}%
  \else
    % gray or color?
    \ifGPcolor
      \def\colorrgb#1{\color[rgb]{#1}}%
      \def\colorgray#1{\color[gray]{#1}}%
      \expandafter\def\csname LTw\endcsname{\color{white}}%
      \expandafter\def\csname LTb\endcsname{\color{black}}%
      \expandafter\def\csname LTa\endcsname{\color{black}}%
      \expandafter\def\csname LT0\endcsname{\color[rgb]{1,0,0}}%
      \expandafter\def\csname LT1\endcsname{\color[rgb]{0,1,0}}%
      \expandafter\def\csname LT2\endcsname{\color[rgb]{0,0,1}}%
      \expandafter\def\csname LT3\endcsname{\color[rgb]{1,0,1}}%
      \expandafter\def\csname LT4\endcsname{\color[rgb]{0,1,1}}%
      \expandafter\def\csname LT5\endcsname{\color[rgb]{1,1,0}}%
      \expandafter\def\csname LT6\endcsname{\color[rgb]{0,0,0}}%
      \expandafter\def\csname LT7\endcsname{\color[rgb]{1,0.3,0}}%
      \expandafter\def\csname LT8\endcsname{\color[rgb]{0.5,0.5,0.5}}%
    \else
      % gray
      \def\colorrgb#1{\color{black}}%
      \def\colorgray#1{\color[gray]{#1}}%
      \expandafter\def\csname LTw\endcsname{\color{white}}%
      \expandafter\def\csname LTb\endcsname{\color{black}}%
      \expandafter\def\csname LTa\endcsname{\color{black}}%
      \expandafter\def\csname LT0\endcsname{\color{black}}%
      \expandafter\def\csname LT1\endcsname{\color{black}}%
      \expandafter\def\csname LT2\endcsname{\color{black}}%
      \expandafter\def\csname LT3\endcsname{\color{black}}%
      \expandafter\def\csname LT4\endcsname{\color{black}}%
      \expandafter\def\csname LT5\endcsname{\color{black}}%
      \expandafter\def\csname LT6\endcsname{\color{black}}%
      \expandafter\def\csname LT7\endcsname{\color{black}}%
      \expandafter\def\csname LT8\endcsname{\color{black}}%
    \fi
  \fi
  \setlength{\unitlength}{0.0500bp}%
  \begin{picture}(7200.00,5040.00)%
    \gplgaddtomacro\gplbacktext{%
      \csname LTb\endcsname%
      \put(726,440){\makebox(0,0)[r]{\strut{}-40}}%
      \put(726,922){\makebox(0,0)[r]{\strut{}-20}}%
      \put(726,1403){\makebox(0,0)[r]{\strut{} 0}}%
      \put(726,1885){\makebox(0,0)[r]{\strut{} 20}}%
      \put(726,2367){\makebox(0,0)[r]{\strut{} 40}}%
      \put(726,2848){\makebox(0,0)[r]{\strut{} 60}}%
      \put(726,3330){\makebox(0,0)[r]{\strut{} 80}}%
      \put(726,3812){\makebox(0,0)[r]{\strut{} 100}}%
      \put(726,4293){\makebox(0,0)[r]{\strut{} 120}}%
      \put(726,4775){\makebox(0,0)[r]{\strut{} 140}}%
      \put(858,220){\makebox(0,0){\strut{} 0}}%
      \put(2047,220){\makebox(0,0){\strut{} 0.5}}%
      \put(3236,220){\makebox(0,0){\strut{} 1}}%
      \put(4425,220){\makebox(0,0){\strut{} 1.5}}%
      \put(5614,220){\makebox(0,0){\strut{} 2}}%
      \put(6803,220){\makebox(0,0){\strut{} 2.5}}%
    }%
    \gplgaddtomacro\gplfronttext{%
      \csname LTb\endcsname%
      \put(5816,4602){\makebox(0,0)[r]{\strut{}"traj.dat" u 1:2}}%
      \csname LTb\endcsname%
      \put(5816,4382){\makebox(0,0)[r]{\strut{}"" u 1:3}}%
      \csname LTb\endcsname%
      \put(5816,4162){\makebox(0,0)[r]{\strut{}"" u 1:8}}%
      \csname LTb\endcsname%
      \put(5816,3942){\makebox(0,0)[r]{\strut{}"" u 1:9}}%
    }%
    \gplbacktext
    \put(0,0){\includegraphics{tunnel01}}%
    \gplfronttext
  \end{picture}%
\endgroup

  \caption{increasing the strength of coupling allows for tunneling. the tunneling doesn't happen till $n1\to n2$}
  \label{fig:tunnel01}
\end{figure}

\subsection{only one exponential}
Thu Aug 29 14:16:47 PDT 2013
\begin{figure}[H]
  \centering
  % GNUPLOT: LaTeX picture with Postscript
\begingroup
  \makeatletter
  \providecommand\color[2][]{%
    \GenericError{(gnuplot) \space\space\space\@spaces}{%
      Package color not loaded in conjunction with
      terminal option `colourtext'%
    }{See the gnuplot documentation for explanation.%
    }{Either use 'blacktext' in gnuplot or load the package
      color.sty in LaTeX.}%
    \renewcommand\color[2][]{}%
  }%
  \providecommand\includegraphics[2][]{%
    \GenericError{(gnuplot) \space\space\space\@spaces}{%
      Package graphicx or graphics not loaded%
    }{See the gnuplot documentation for explanation.%
    }{The gnuplot epslatex terminal needs graphicx.sty or graphics.sty.}%
    \renewcommand\includegraphics[2][]{}%
  }%
  \providecommand\rotatebox[2]{#2}%
  \@ifundefined{ifGPcolor}{%
    \newif\ifGPcolor
    \GPcolorfalse
  }{}%
  \@ifundefined{ifGPblacktext}{%
    \newif\ifGPblacktext
    \GPblacktexttrue
  }{}%
  % define a \g@addto@macro without @ in the name:
  \let\gplgaddtomacro\g@addto@macro
  % define empty templates for all commands taking text:
  \gdef\gplbacktext{}%
  \gdef\gplfronttext{}%
  \makeatother
  \ifGPblacktext
    % no textcolor at all
    \def\colorrgb#1{}%
    \def\colorgray#1{}%
  \else
    % gray or color?
    \ifGPcolor
      \def\colorrgb#1{\color[rgb]{#1}}%
      \def\colorgray#1{\color[gray]{#1}}%
      \expandafter\def\csname LTw\endcsname{\color{white}}%
      \expandafter\def\csname LTb\endcsname{\color{black}}%
      \expandafter\def\csname LTa\endcsname{\color{black}}%
      \expandafter\def\csname LT0\endcsname{\color[rgb]{1,0,0}}%
      \expandafter\def\csname LT1\endcsname{\color[rgb]{0,1,0}}%
      \expandafter\def\csname LT2\endcsname{\color[rgb]{0,0,1}}%
      \expandafter\def\csname LT3\endcsname{\color[rgb]{1,0,1}}%
      \expandafter\def\csname LT4\endcsname{\color[rgb]{0,1,1}}%
      \expandafter\def\csname LT5\endcsname{\color[rgb]{1,1,0}}%
      \expandafter\def\csname LT6\endcsname{\color[rgb]{0,0,0}}%
      \expandafter\def\csname LT7\endcsname{\color[rgb]{1,0.3,0}}%
      \expandafter\def\csname LT8\endcsname{\color[rgb]{0.5,0.5,0.5}}%
    \else
      % gray
      \def\colorrgb#1{\color{black}}%
      \def\colorgray#1{\color[gray]{#1}}%
      \expandafter\def\csname LTw\endcsname{\color{white}}%
      \expandafter\def\csname LTb\endcsname{\color{black}}%
      \expandafter\def\csname LTa\endcsname{\color{black}}%
      \expandafter\def\csname LT0\endcsname{\color{black}}%
      \expandafter\def\csname LT1\endcsname{\color{black}}%
      \expandafter\def\csname LT2\endcsname{\color{black}}%
      \expandafter\def\csname LT3\endcsname{\color{black}}%
      \expandafter\def\csname LT4\endcsname{\color{black}}%
      \expandafter\def\csname LT5\endcsname{\color{black}}%
      \expandafter\def\csname LT6\endcsname{\color{black}}%
      \expandafter\def\csname LT7\endcsname{\color{black}}%
      \expandafter\def\csname LT8\endcsname{\color{black}}%
    \fi
  \fi
  \setlength{\unitlength}{0.0500bp}%
  \begin{picture}(7200.00,5040.00)%
    \gplgaddtomacro\gplbacktext{%
      \csname LTb\endcsname%
      \put(594,440){\makebox(0,0)[r]{\strut{}-15}}%
      \put(594,1163){\makebox(0,0)[r]{\strut{}-10}}%
      \put(594,1885){\makebox(0,0)[r]{\strut{}-5}}%
      \put(594,2608){\makebox(0,0)[r]{\strut{} 0}}%
      \put(594,3330){\makebox(0,0)[r]{\strut{} 5}}%
      \put(594,4053){\makebox(0,0)[r]{\strut{} 10}}%
      \put(594,4775){\makebox(0,0)[r]{\strut{} 15}}%
      \put(726,220){\makebox(0,0){\strut{} 0}}%
      \put(1334,220){\makebox(0,0){\strut{} 0.2}}%
      \put(1941,220){\makebox(0,0){\strut{} 0.4}}%
      \put(2549,220){\makebox(0,0){\strut{} 0.6}}%
      \put(3157,220){\makebox(0,0){\strut{} 0.8}}%
      \put(3765,220){\makebox(0,0){\strut{} 1}}%
      \put(4372,220){\makebox(0,0){\strut{} 1.2}}%
      \put(4980,220){\makebox(0,0){\strut{} 1.4}}%
      \put(5588,220){\makebox(0,0){\strut{} 1.6}}%
      \put(6195,220){\makebox(0,0){\strut{} 1.8}}%
      \put(6803,220){\makebox(0,0){\strut{} 2}}%
    }%
    \gplgaddtomacro\gplfronttext{%
      \csname LTb\endcsname%
      \put(5816,4602){\makebox(0,0)[r]{\strut{}var2}}%
      \csname LTb\endcsname%
      \put(5816,4382){\makebox(0,0)[r]{\strut{}var3}}%
      \csname LTb\endcsname%
      \put(5816,4162){\makebox(0,0)[r]{\strut{}var4}}%
      \csname LTb\endcsname%
      \put(5816,3942){\makebox(0,0)[r]{\strut{}var5}}%
      \csname LTb\endcsname%
      \put(5816,3722){\makebox(0,0)[r]{\strut{}var6}}%
      \csname LTb\endcsname%
      \put(5816,3502){\makebox(0,0)[r]{\strut{}var7}}%
      \csname LTb\endcsname%
      \put(5816,3282){\makebox(0,0)[r]{\strut{}var8}}%
      \csname LTb\endcsname%
      \put(5816,3062){\makebox(0,0)[r]{\strut{}var9}}%
    }%
    \gplbacktext
    \put(0,0){\includegraphics{oneexp}}%
    \gplfronttext
  \end{picture}%
\endgroup

  \caption{a potential with only one exponential. the proton turns back and $p1$ and $x1$ are oscillating}
  \label{fig:oneexp}
\end{figure}
\end{document}
